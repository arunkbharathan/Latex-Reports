% $Log: abstract.tex,v $
% Revision 1.1  93/05/14  14:56:25  starflt
% Initial revision
% 
% Revision 1.1  90/05/04  10:41:01  lwvanels
% Initial revision
% 
%
%% The text of your abstract and nothing else (other than comments) goes here.
%% It will be single-spaced and the rest of the text that is supposed to go on
%% the abstract page will be generated by the abstractpage environment.  This
%% file should be \input (not \include 'd) from cover.tex.
\begin{center}{\large{\bf {Abstract}}}
\end{center}
\noindent
Brain computer interfaces (BCIs) have the potential to offer humans a new and innovative nonmuscular modality through which to communicate directly via their brain activity with the environment. These systems rely on the acquisition and interpretation of the commands encoded in neurophysiological signals without using the conventional muscular output pathways of the central nervous system (CNS). Brain imaging technologies such as EEG, fMRI and MEG are used to observe this neurophysiological activity. Electroencephalograph (EEG) is the only practical noninvasive, cheap and real-time capable imaging technology for use in a BCI system. BCIs propose to offer people who suffer from neuromuscular disorders, whom lack any voluntary motor movement, with the only possibility of communication and control.
              
This thesis firstly addresses the issues for using EEG as a BCI input modality by reviewing the methods for EEG acquisition and analysis. The components and methodologies for a BCI system framework and the state of the art in this technology are then presented. Feature extraction and classification are the main stages in the BCI system. The feature extraction stage identifies discriminative information in the brain signals that have been recorded. The classification stage classifies the signals by taking the feature vectors into account.

A one step process is used which combines the discrimination and classification stages with a linear classifier employing a regularization scheme based on the spectral $\ell $1-norm and Tikhonov regularization of its coefficient matrix. The spectral regularization not only provides a principled way of complexity control but also enables automatic determination of the rank of the coefficient matrix. Using the Linear Matrix Inequality technique, we formulate the inference task as a single convex optimization problem. This method is applied to the motor-imagery EEG classification problem. The method not only improves upon conventional methods in the classification performance but also determines a subspace in the signal that concentrates discriminative information without any additional feature extraction step.



