%% This defines the bibliography file (main.bib) and the bibliography style.
%% If you want to create a bibliography file by hand, change the contents of
%% this file to a `thebibliography' environment.  For more information 
%% see section 4.3 of the LaTeX manual.
%\begin{singlespace}
%\bibliography{main}
%\bibliographystyle{plain}
%\end{singlespace}
\begin{thebibliography}{99}

\bibitem{1}Graimann, Bernhard, Brendan Allison, and Gert Pfurtscheller. "Brain–computer interfaces: A gentle introduction." In Brain-Computer Interfaces, pp. 1-27. Springer Berlin Heidelberg, 2010.
\bibitem{2}Nicolas-Alonso, Luis Fernando, and Jaime Gomez-Gil. "Brain computer interfaces, a review." Sensors 12, no. 2 (2012): 1211-1275
\bibitem{3}Pfurtscheller, Gert, and F. H. Lopes da Silva. "Event-related EEG/MEG synchronization and desynchronization: basic principles." Clinical neurophysiology 110, no. 11 (1999): 1842-1857.
\bibitem{4}Yang, B. H., G. Z. Yan, and R. G. Yan. "[A review of brain-computer interfaces (BCIs)]." Zhongguo yi liao qi xie za zhi= Chinese journal of medical instrumentation 29, no. 5 (2005): 353. 
\bibitem{5}Makeig, Scott, Stefan Debener, Julie Onton, and Arnaud Delorme. "Mining event-related brain dynamics." Trends in cognitive sciences 8, 
no. 5 (2004): 204-210.
\bibitem{6}Musallam, Sam, B. D. Corneil, Bradley Greger, Hans Scherberger, and R. A. Andersen. "Cognitive control signals for neural prosthetics." Science 305, no. 5681 (2004): 258-262.
\bibitem{7}Blankertz, Benjamin, Steven Lemm, Matthias Treder, Stefan Haufe, and Klaus-Robert Müller. "Single-trial analysis and classification of ERP components—a tutorial." NeuroImage 56, no. 2 (2011): 814-825.
\bibitem{8}Lotte, Fabien, Marco Congedo, Anatole Lécuyer, Fabrice Lamarche, and Bruno Arnaldi. "A review of classification algorithms for EEG-based  rain–computer interfaces." Journal of neural engineering 4 (2007).
\bibitem{9}Bashashati, Ali, Mehrdad Fatourechi, Rabab K. Ward, and Gary E. Birch. "A survey of signal processing algorithms in brain–computer interfaces based on electrical brain signals." Journal of Neural engineering 4, no. 2 (2007): R32

\bibitem {10} Blankertz, Benjamin, Ryota Tomioka, Steven Lemm, Motoaki Kawanabe, and K-R. Muller. "Optimizing spatial filters for robust EEG single-trial analysis." Signal Processing Magazine, IEEE 25, no. 1 (2008): 41-56.
\bibitem {11}  Blankertz, Benjamin, Guido Dornhege, Matthias Krauledat, Klaus-Robert Müller, and Gabriel Curio. "The non-invasive Berlin brain–computer interface: fast acquisition of effective performance in untrained subjects." NeuroImage 37, no. 2 (2007): 539-550.
\bibitem {12}Fukunaga, Keinosuke. Introduction to statistical pattern recognition. Access Online via Elsevier, 1990.
\bibitem{13}Ramoser, Herbert, Johannes Muller-Gerking, and Gert Pfurtscheller. "Optimal spatial filtering of single trial EEG during imagined hand movement." Rehabilitation Engineering, IEEE Transactions on 8, no. 4 (2000): 441-446.	
\bibitem {14} Tomioka, Ryota, and Klaus-Robert Müller. "A regularized discriminative framework for EEG analysis with 
application to brain–computer interface." Neuroimage 49, no. 1 (2010): 415-432.
\bibitem {15}  Ang, Kai Keng, Zhang Yang Chin, Haihong Zhang, and Cuntai Guan. "Filter bank common spatial pattern (FBCSP) in brain-computer interface." In Neural Networks, 2008. IJCNN 2008.(IEEE World Congress on Computational Intelligence). IEEE International Joint Conference on, pp. 2390-2397. IEEE, 2008.
\bibitem {16}Tomioka, Ryota, Guido Dornhege, Kazuyuki Aihara, and K-R. Müller. "An iterative algorithm for spatio-temporal filter optimization." In Verlag der Technischen Universität Graz. 2006. 
\bibitem{17} Tomioka, Ryota, Guido Dornhege, Guido Nolte, Kazuyuki Aihara, and Klaus-Robert Müller. "Optimizing spectral filters for single trial EEG classification." In Pattern Recognition, pp. 414-423. Springer Berlin Heidelberg, 2006.
\bibitem{18}Melzer, Thomas. "SVD and its application to generalized eigenvalue problems." Vienna University of Technology (2004).
\bibitem {19}Lewis, Adrian S., and Michael L. Overton. "Eigenvalue optimization." Acta numerica 5, no. 1 (1996): 149-190.
\bibitem {20} Boyd, Stephen Poythress, and Lieven Vandenberghe. Convex optimization. Cambridge university press, 2004.
\bibitem {21}  Vandenberghe, Lieven, and Stephen Boyd. "Semidefinite programming." SIAM review 38, no. 1 (1996): 
49-95.
\bibitem {22} Fisher, Ronald A. "The use of multiple measurements in taxonomic problems." Annals of eugenics 7, no. 
2 (1936): 179-188.
\bibitem {23} Welling, Max. "Fisher linear discriminant analysis." Department of Computer Science, University of Toronto (2005).

\bibitem {24} Tomioka, Ryota, Kazuyuki Aihara, and Klaus-Robert Müller. "Logistic regression for single trial EEG 
classification." Advances in neural information processing systems 19 (2007): 1377-1384.
\bibitem {25}Srebro, Nathan, and Adi Shraibman. "Rank, trace-norm and max-norm." In Learning Theory, pp. 545-560. 
Springer Berlin Heidelberg, 2005.
\bibitem {26}Rennie, Jason DM. "The Relation Between the Spectral and Trace Norms." (2006).
\bibitem {27}Fazel, Maryam, Haitham Hindi, and Stephen P. Boyd. "A rank minimization heuristic with application to 
minimum order system approximation." In American Control Conference, 2001. Proceedings of the 2001, vol. 6, pp. 
4734-4739. IEEE, 2001.
\bibitem {28}Grant, Michael, Stephen Boyd, and Yinyu Ye. "CVX: Matlab software for disciplined convex programming." 
(2008).
\bibitem {29}Ryota Tomioka  Masashi Sugiyama, "Dual Augmented Lagrangian Method for Efficient Sparse 
Reconstruction", IEEE Signal Proccesing Letters, 16 (12) pp. 1067-1070, 2009.


\bibitem {30}Delorme, Arnaud, Tim Mullen, Christian Kothe, Zeynep Akalin Acar, Nima Bigdely-Shamlo, Andrey Vankov, and Scott Makeig. "EEGLAB, SIFT, NFT, BCILAB, and ERICA: new tools for advanced EEG processing." Computational intelligence and neuroscience 2011 (2011): 10.
\bibitem {31}	Data sets 2a: ‹4-class motor imagery› (description)
provided by the Institute for Knowledge Discovery (Laboratory of Brain-Computer Interfaces), Graz University of Technology, (Clemens Brunner, Robert Leeb, Gernot Müller-Putz, Alois Schlögl, Gert Pfurtscheller)
\bibitem {32} Data set I: ‹motor imagery in ECoG recordings, session-to-session transfer› (description I) provided by Eberhard-Karls-Universität Tübingen, Germany, Dept. of Computer Engineering and Dept. of Medical Psychology and Behavioral Neurobiology (Niels Birbaumer), and Max-Planck-Institute for Biological Cybernetics, Tübingen, Germany (Bernhard Schökopf), and Universität Bonn, Germany, Dept. of Epileptology 
\bibitem {33}  He, P., G. Wilson, and C. Russell. "Removal of ocular artifacts from electro-encephalogram by adaptive filtering." Medical and biological engineering and computing 42, no. 3 (2004): 407-412.
\bibitem {34} Lotte, Fabien, and Cuntai Guan. "Regularizing common spatial patterns to improve BCI designs: unified theory and new algorithms." Biomedical Engineering, IEEE Transactions on 58, no. 2 (2011): 355-362.
\bibitem {35}Nash, Stephen G., and Jorge Nocedal. "A numerical study of the limited memory BFGS method and the truncated-Newton method for large scale optimization." SIAM Journal on Optimization 1, no. 3 (1991): 358-372.
\end{thebibliography}


%Shields G. and Walton G. (1998): ‘Cite them Right! How to Organise a Report’ [Online], Available HTTP: http://www.unn.ac.uk/central/isd/cite/ [22 Feb 01].
