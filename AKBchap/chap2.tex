\chapter{Introduction To Brain Computer Interfaces}


\section{Introduction}\label{ch2:1}

Brain Computer Interface (BCIs)~\cite{1} started with Hans Berger's inventing of electrical activity of the human brain and the development of electroencephalography (EEG). In 1924 Berger recorded an EEG signals from a human brain for the first time. By analyzing EEG signals Berger was able to identify oscillatory activity in the brain, such as the alpha wave $8-12$ Hz, also known as Berger's wave. The first recording device used by Berger was very elementary, which was in the early stages of development and was required to insert silver wires under the scalp of the patients. In later stages, those were replaced by silver foils that were attached to the patients head by rubber bandages. More sophisticated measuring devices such as the Siemens double-coil recording galvanometer, which displayed electric voltages as small as one ten thousandth of a volt, led to success. Berger analyzed the interrelation of alternations in his EEG wave diagrams with brain diseases. EEGs permitted completely new possibilities for the research of human brain activities.

In this chapter we review the aspects of BCI research mentioned above and highlight recent developments and open problems. The review is ordered by the steps that are needed for brain computer communication. We start with methods for measuring brain activity (Section 2.3) and then give a description of the neurophysiologic signals that can be used in BCI systems (Section 2.4).  The translation of signals into commands with the help of signal processing and classification methods is described in Section 2.5. Finally, applications that can be controlled with a BCI are described in Section 2.6.

\section{Overview of Brain Computer Interfaces}\label{ch2:2}     
          
Any natural form of communication or control requires peripheral nerves and muscles. The process begins with the user's intent. This intent triggers a complex process in which certain brain areas are activated, and hence signals are sent via the peripheral nervous system to the corresponding muscles, which in turn perform the movement necessary for the communication or control task. The activity resulting from this process is often called motor output or efferent output. Efferent means conveying signal from central to peripheral nervous system and further to an effecter such as muscle. The efferent pathway is necessary for motor control, the reverse of it afferent pathway (sensory pathway) is for learning motor skills and dexterous tasks, such as typing or playing a musical instrument. But, a BCI offers alternate pathway to natural communication and control, a man made system that bypasses body's normal path way from central nervous system. Instead of depending on peripheral nervous system and muscles, a BCI directly measures brain activity associated with an action from user and translates this brain activity into corresponding control signals. This translation involves signal processing and pattern recognition which is done by a computer. Since the measured activity originates directly from the brain and not from the peripheral systems or muscles, the system is called a Brain Computer Interface~\cite{2}. 

A Brain Computer Interface (BCI) provides a communication path between human brain and the computer system~\cite{3}. The major goal of BCI research is to develop a system that allows disabled people to communicate with other persons and helps to interact with the external environments. This area includes components like, comparison of invasive and non invasive technologies to measure brain activity, evaluation of control signals (i.e. patterns of brain activity that can be used for communication), development of algorithms for translation of brain signals into computer commands, and the development of new BCI applications.

A BCI is an artificial intelligence system that can recognize a certain set of patterns in brain signals following five consecutive stages: signal acquisition, pre-processing or signal enhancement, feature extraction, classification, and the control interface. The signal acquisition stage captures the brain signals and may also perform noise reduction and artifact processing. The pre-processing stage prepares the signals in a suitable form for further processing. The feature extraction stage identifies discriminative information in the brain signals that have been recorded. Once measured, the signal is mapped onto a vector containing effective and discriminant features from the observed signals. The extraction of this interesting information is a very challenging task. Brain signals are mixed with other signals coming from a finite set of brain activities that overlap in both time and space. Moreover, the signal is not usually stationary and may also be distorted by artifacts such as electromyography (EMG) or electrooculography (EOG). The feature vector must also be of a low dimension, in order to reduce feature extraction stage complexity, but without relevant information loss. The classification stage classifies the signals taking the feature vectors into account. The choice of good discriminative features is therefore essential to achieve effective pattern recognition, in order to decipher the user's intentions. Finally the control interface stage translates the classified signals into meaningful commands for any connected device, such as a wheelchair or a computer.


\section{Types of BCIs}\label{ch2:3}
  
The BCIs can be categorized into 
\begin{enumerate}
\item exogenous or endogenous
\item synchronous (cue-paced) or asynchronous (self-paced)
\end{enumerate}
According to the nature of the signals used as input, BCI systems can be classified as either exogenous or endogenous. Exogenous BCI uses the neuron activity elicited in the brain by an external stimulus such as visual or auditory evoked potentials. Exogenous systems do not require extensive training since their control signals, SSVEPs and P300, can be easily and quickly set-up. Besides, the signal controls can be realized with only one EEG channel and can achieve a high information transfer rate of up to 60 bits/min. On the other hand, endogenous BCI is based on self-regulation of brain rhythms and potentials without external stimuli. Through neurofeedback training, the users learn to generate specific brain patterns which may be decoded by the BCI such as modulations in the sensorimotor rhythms or the Slow Cortical Potentials. The advantage of an endogenous BCI is that the user can operate the BCI at free will and move a cursor to any point in a two-dimensional space, while an exogenous BCI may constrain the user to the choices presented. Also, endogenous BCI are especially useful for users with advanced stages of ALS or whose sensory organs are affected~\cite{4}.

According to the input data processing modality, BCI systems can be classified as synchronous or asynchronous. Synchronous BCIs analyze brain signals during predefined time windows. Any brain signal outside the predefined window is ignored. Therefore, the user is only allowed to send commands during specific periods determined by the BCI system. For example, the standard Graz BCI represents a synchronous BCI system. The advantage of a synchronous BCI system is that the onset of mental activity is known in advance and associated with a specific cue. Moreover, the patients may also perform blinks and other eye movements, which would generate artifacts, if the BCI did not analyze the brain signals to avoid their misleading effects. This simplifies the design and evaluation of synchronous BCI. Asynchronous BCIs continuously analyze brain signals no matter when the user acts. They offer a more natural mode of human-machine interaction than synchronous BCI. However, asynchronous BCIs are more computation demanding and complex. 

\section{Signal Acquisition-Neuro imaging techniques}\label{ch2:4}

BCIs use brain signals to gather information on user intentions.. Two types of brain activities~\cite{5} may be monitored: 

\begin{enumerate}
\item electrophysiological
\item hemodynamic
\end{enumerate}

Electrophysiological activity is generated by electro-chemical transmitters exchanging information between the neurons. Electrophysiological activity is measured by electroencephalography, electrocorticography, magnetoencephalography, and electrical signal acquisition in single neurons. The hemodynamic response is a process in which the blood releases glucose to active neurons at a greater rate than in the area of inactive neurons. The glucose and oxygen delivered through the blood stream results in a surplus of oxyhemoglobin in the veins of the active area, and in a distinguishable change of the local ratio of oxyhemoglobin to deoxyhemoglobin. These changes can be quantified by neuroimaging methods such as functional magnetic resonance and near infrared spectroscopy. These kinds of methods are categorized as indirect, because they measure the hemodynamic response, which, in contrast to electrophysiological activity, is not directly related to neuronal activity~\cite{2}.

\subsection{Electroencephalogram}\label{ch2:5}

EEG measures electric brain activity caused by the flow of electric currents during synaptic excitations of the dendrites in the neurons. EEG signals are easily recorded in a non-invasive manner through electrodes placed on the scalp, for that reason it is by far the most widespread recording modality. However, it provides very poor quality signals as the signals have to cross the scalp, skull, and many other layers. This means that EEG signals in the electrodes are weak, and of poor spatial resolution. This technique is moreover severely affected by background noise generated either inside the brain or externally over the scalp.

The EEG recording system consists of electrodes, amplifiers, A/D converter, and a recording device. The electrodes acquire the signal from the scalp, the amplifiers process the analogue signal to enlarge the amplitude of the EEG signals so that the A/D converter can digitalize the signal in a more accurate way. Finally, the recording device, which may be a personal computer or similar, stores, and displays the data.
    
A minimal configuration for EEG measurement therefore consists of one measurement, one reference, and one ground electrode. Multi-channel configurations can comprise up to $256$ measurement electrodes. These electrodes are usually made of silver chloride (AgCl). Electrode-scalp contact impedance should be between $1 k\Omega$ and $10 k\Omega$ to record signal accurately. The electrode-tissue interface is not only resistive but also capacitive and it therefore behaves as a low pass filter. EEG gel creates a conductive path between the skin and each electrode that reduces the impedance. Use of the gel is cumbersome, however, as continued maintenance is required to assure a relatively good quality signal. Electrodes that do not need to use gels, called 'dry' electrodes, have been made with other materials such as titanium and stainless-steel.
 
EEG comprises a set of signals which may be classified according to their frequency as shown in Fig.~\ref{Brhym}. According to the distribution over the scalp or biological significance, well-known frequency bands have been defined. These frequency bands are referred to as delta ($\delta$), theta ($\theta$), alpha ($\alpha$), beta ($\beta$), and gamma ($\gamma$) from low to high, respectively.
\begin{figure}
\centering
\includegraphics[scale=0.8]{Figure1/brainrhythm1.png}
\caption{Visualization of Brain Rhythms}
\label{Brhym}
\end{figure}
The delta ($\delta$) band lies below $4$ Hz, delta rhythms are usually only observed in adults in deep sleep state and are unusual in adults in an awake state. Due to low frequency, it is easy to confuse delta waves with artifact signals, which are caused by the large muscles of the neck or jaw.

Theta ($\theta$) waves lie within the $4$ to $7$ Hz range. Theta band has been associated with meditative concentration and a wide range of cognitive processes such as mental calculation, maze task demands, or conscious awareness.

Alpha ($\alpha$) rhythms are found over the occipital region in the brain. These waves lie within the $8$ to $12$ Hz range. Their amplitude increases when the eyes close and the body relaxes and they attenuate when the eyes open and mental effort is made. These rhythms primarily reflect visual processing in the occipital brain region and may also be related to the memory brain function, that alpha activity may be associated with mental effort. Increasing mental effort causes a suppression of alpha activity, particularly from the frontal areas. Mu rhythms may be found in the same range as alpha rhythms, mu rhythms are strongly connected to motor activities.

Beta ($\beta$) rhythms, within the $12$ to $30$ Hz range, are recorded in the frontal and central regions of the brain and are associated with motor activities. Beta rhythms are desynchronized during real movement or motor imagery. Beta waves are characterized by their symmetrical distribution when there is no motor activity. However, in case of active movement, the beta waves attenuate, and their symmetrical distribution changes.

Gamma ($\gamma$) rhythms belong to the frequency range from $30$ to $100$ Hz. The presence of gamma waves in the brain activity of a healthy adult is related to certain motor functions or perceptions, among others Gamma rhythms are less commonly used in EEG-based BCI systems, because artifacts such as electromyography (EMG) or electrooculography (EOG) are likely to affect them. Nevertheless, this range is attracting growing attention in BCI research because, compared to traditional beta and alpha signals, gamma activity may increase the information transfer rate and offer higher spatial resolution.

\subsection{Electrocorticogram}\label{ch2:5}

ECoG is a technique that measures electrical activity in the cerebral cortex by means of electrodes placed directly on the surface of the brain. Compared to EEG, ECoG provides higher temporal and spatial resolution as well as higher amplitudes and a lower vulnerability to artifacts such as blinks and eye movement. However, ECoG is an invasive recording modality which requires a craniotomy to implant an electrode grid, entailing significant health hazards.  ECoG has been used for the analysis of alpha and beta waves or gamma waves produced during voluntary motor action.

\subsection{Magnetoencephalography (MEG)}\label{ch2:5}

MEG is a non-invasive imaging technique that registers the brain’s magnetic activity by means of magnetic induction. MEG measures the intracellular currents flowing through dendrites which produce magnetic fields that are measurable outside of the head. The neurophysiological processes that produce MEG signals are identical to those that produce EEG signals. The advantage of MEG is that magnetic fields are less distorted by the skull and scalp than electric fields. Magnetic fields are detected by superconducting quantum interference devices, which are extremely sensitive to magnetic disturbances produced by neural activity. MEG requires effective shielding from electromagnetic interferences.MEG provides signals with higher spatiotemporal resolution than EEG, which reduces the training time needed to control a BCI and speeds up reliable communications. MEG has also been successfully used to localize active regions inside the brain. In spite of these advantageous features, MEG is not often used in BCI design because MEG technology is too bulky and expensive to become an acquisition modality suitable for everyday use.

\subsection{Functional Magnetic Resonance Imaging (fMRI)}\label{ch2:5}

fMRI is a non-invasive neuroimaging technique which detects changes in local cerebral blood volume, cerebral blood flow and oxygenation levels during neural activation by means of electromagnetic fields. fMRI is generally performed using MRI scanners which apply electromagnetic fields of strength in the order of $3T$ or $7T$. The main advantage of the use of fMRI is high space resolution. For that reason, fMRI have been applied for localizing active regions inside the brain. However, fMRI has a low temporal resolution of about $1$ or $2$ seconds. Additionally, the hemodynamic response introduces a physiological delay from $3$ to $6$ seconds. fMRI appears unsuitable for rapid communication in BCI systems and is highly susceptible to head motion artifacts. fMRI requires overly bulky and expensive hardware.

\subsection{Near Infrared Spectroscopy (NIRS)}\label{ch2:5}

NIRS is an optical spectroscopy method that employs infrared light to characterize noninvasively acquired fluctuations in cerebral metabolism during neural activity. Infrared light penetrates the skull to a depth of approximately 1–3 cm below its surface, where the intensity of the attenuated light allows alterations in oxyhemoglobin and deoxyhemoglobin concentrations to be measured. Due to shallow light penetration in the brain, this optical neuroimaging technique is limited to the outer cortical layer. In a similar way to fMRI, one of the major limitations of NIRS is the nature of the hemodynamic response, because vascular changes occur a certain number of seconds after its associated neural activity. The spatial resolution of NIRS is quite low, in the order of 1 cm. Nevertheless, NIRS offers low cost, high portability, and an acceptable temporal resolution in the order of 100 milliseconds.

\section{Control Signal Types in BCIs}\label{ch2:5}

The purpose of a BCI is to interpret user intentions by means of monitoring cerebral activity. Brain signals involve numerous simultaneous phenomena related to cognitive tasks. Most of them are still incomprehensible and their origins are unknown. However, the physiological phenomena of some brain signals have been decoded in such way that people may learn to modulate them at will, to enable the BCI systems to interpret their intentions. These signals are regarded as possible control signals in BCIs.

Numerous studies have described a vast group of brain signals that might serve as control signals in BCI systems. The control signals that are employed in current BCI systems such as visual evoked potentials, slow cortical potentials P300 evoked potentials, and sensorimotor rhythms are discussed~\cite{6}.

\subsection{Visual Evoked Potentials (VEPs)}\label{ch2:5}

VEPs are brain activity modulations that occur in the visual cortex after receiving a visual stimulus~\cite{7}. These modulations are relatively easy to detect since the amplitude of VEPs increases enormously as the stimulus is moved closer to the central visual field. VEPs may be classified according to three different criteria: (i) by the morphology of the optical stimuli, (ii) by the frequency of visual stimulation; and (iii) by field stimulation. According to the first criterion, VEPs may be caused by using flash stimulation or using graphic patterns such as checkerboard lattice, gate, and random-dot map. According to the frequency, VEPs can also be classified as transient VEPs (TVEPs) and as steady-state VEPs (SSVEPs). TVEPs occur when the frequency of visual stimulation is below 6 Hz, while SSVEPs occur in reaction to stimuli of a higher frequency. Lastly, according to the third criterion, VEPs can be divided into whole field VEPs, half field VEPs, and part field VEPs depending on the area of on-screen stimulus. For instance, if only half of the screen displays graphics, the other half will not display any visual stimulation and the person will look at the centre of the screen, which will induce a half  field VEP. SSVEP-based BCIs allow users to select a target by means of an eye-gaze. The user visually fixes attention on a target and the BCI identifies the target through SSVEP features analysis.

\subsection{Slow Cortical Potentials (SCPs)}\label{ch2:5}

SCPs are slow voltage shifts in the EEG that last a second to several seconds. SCPs belong to the part of the EEG signals below 1 Hz. SCPs are associated with changes in the level of cortical activity. Negative SCPs correlate with increased neuronal activity, whereas positive SCPs coincide with decreased activity in individual cells. These brain signals can be self-regulated by both healthy users and paralyzed patients to control external devices by means of a BCI. SCP shifts can be used to move a cursor and select the targets presented on a computer screen.

People can be trained to generate voluntary SCP changes using a thought-translation device. The thought-translation device is a tool used for self-regulation SCP training, which shows visual-auditory marks so that the user can learn to shift the SCP. The thought-translation device typically comprises a cursor on a screen in such a way that the vertical position of the cursor constantly reflects the amplitude of SCP shifts. Success in SCP self-regulation training depends on numerous factors, such as the patient's psychological and physical state, motivation, social context, or the trainer-patient relationship. Therefore, the value of SCPs as a suitable control signal for each subject can only be determined on the basis of initial trials. Other factors, such as sleep quality, pain, and mood also have an influence on self-regulation performance. Their effects are not identical for all patients and further investigation is certainly needed to establish general rules on this matter.

\subsection{P300 Evoked Potentials}\label{ch2:5}

P300 evoked potentials are positive peaks in the EEG due to infrequent auditory, visual, or somatosensory stimuli. The P300 responses are elicited about $300 ms$ after attending to an oddball stimulus among several frequent stimuli. Some studies have proven that the less probable the stimulus, the larger the amplitude of the response peak. The use of P300-based BCIs does not require training. However, the performance may be reduced because the user gets used to the infrequent stimulus and consequently P300 amplitude is decreased.

A typical application of a BCI based on visual P300 evoked potentials comprises a matrix of letters, numbers, or other symbols or commands. The rows or columns of this matrix are flashed at random while the EEG is monitored. The user gazes at the desired symbol and counts how many times the row or column containing the desired choice flashes. P300 is elicited only when the desired row or column flashes. Thus, the BCI uses this effect to determine the target symbol. Due to the low signal-to-noise ratio in EEG signals, the detection of target symbols from a single trial is very difficult. The rows or columns must be flashed several times for each choice. The epochs corresponding to each row or column are averaged over the trials, in order to improve their accuracy. However, these repetitions decrease the number of choices per minute, e.g., with $15$ repetitions, only two characters are spelled per minute. Although most of the applications based on P300 evoked potentials employ visual stimuli, auditory stimuli have been used for people with visual impairment.

\subsection{Sensorimotor Rhythms (mu and beta rhythms)}\label{ch2:5}

Sensorimotor rhythms comprise mu and beta rhythms, which are oscillations in the brain activity localized in the mu band $7-13$ Hz, also known as the Rolandic band, and beta band $13-30$ Hz, respectively. Both rhythms are associated in such a way that some beta rhythms are harmonic mu rhythms, although some beta rhythms may also be independent. The amplitude of the sensorimotor rhythms varies when cerebral activity is related to any motor task although actual movement is not required to modulate the amplitude of sensorimotor rhythms. Similar modulation patterns in the motor rhythms are produced as a result of mental rehearsal of a motor act without any motor output. Sensorimotor rhythms have been used to control BCIs, because people can learn to generate these modulations voluntarily in the sensorimotor rhythms. Sensorimotor rhythms can endure two kinds of amplitude modulations known as event-related desynchronization (ERD) and event-related synchronization (ERS) that are generated sensory stimulation, motor behaviour, and mental imagery. ERD involves an amplitude suppression of the rhythm and ERS implies amplitude enhancement. The mu band ERD starts $2.5s$ before movement on-set, reaches the maximal ERD shortly after movement-onset, and recovers its original level within a few seconds. In contrast, the beta rhythm shows a short ERD during the movement initiation of movement, followed by ERS that reaches the maximum after movement execution. This ERS occurs while the mu rhythm is still attenuated. Gamma rhythms reveal an ERS shortly before movement-onset.
 
\section{Features Extraction and Selection}\label{ch2:5}
Different thinking activities result in different patterns of brain signals. BCI is seen as a pattern recognition system that classifies each pattern into a class according to its features. BCI extracts some features from brain signals that reflect similarities to a certain class as well as differences from the rest of the classes. The features are measured or derived from the properties of the signals which contain the discriminative information needed to distinguish their different types~\cite{8}.

\begin{figure}
\centering
\includegraphics[scale=1]{Figure1/bci2.png}
\caption{Two element feature vectors for all exciting trials in red and non-exciting trials in green}
\label{twoele}
\end{figure}

The design of a suitable set of features is a challenging issue. The information of interest in brain signals is hidden in a highly noisy environment, and brain signals comprise a large number of simultaneous sources. A signal that may be of interest could be overlapped in time and space by multiple signals from different brain tasks. For that reason, in many cases, it is not enough to use simple methods such as a band pass filter to extract the desired band power.

Brain signals can be measured through multiples channels. Not all information provided by the measured channels is generally relevant for understanding the underlying phenomena of interest. Dimensionality reduction techniques such as principal component analysis (PCA) or independent component analysis (ICA) can be applied to reduce the dimension of the original data, removing the irrelevant and redundant information. Computational costs are thereby reduced. Brain signals are inherently non-stationary. Time information about when a certain feature occurs should be obtained. Some approaches divide the signals into short segments and the parameters can be estimated from each segment. However, the segment length affects the accuracy of estimated features. FFT performs very poorly with short data segments. Wavelet transform or adaptive autoregressive components are preferred to reveal the non-stationary time variations of brain signals. A two class feature space is shown in Fig.~\ref{twoele}.

Multiples features can be extracted from several channels and from several time segments before being concatenated into a single feature vector. One of the major difficulties in BCI design is choosing relevant features from the vast number of possible features. High dimensional feature vectors are not desirable due to the "curse of dimensionality" in training classification algorithms. The feature selection may be attempted examining all possible subsets of the features. However, the number of possibilities grows exponentially, making an exhaustive search impractical for even a moderate number of features. Some more efficient optimization algorithms can be applied with the aim of minimizing the number of features while maximizing the classification performance.

For dimensionality reduction principle Component analysis or Independent Component analysis are used. AutoRegressive Components (AR), Matched Filtering (MF), Wavelet Transform (WT), Common Spatial Pattern (CSP), Genetic Algorithm (GA), Sequential Selection were used as feature extraction and selection methods. Among them CSP is commonly used and its detailed explanation will be given  in next chapter.


\section {Artifacts in BCIs}

Artifacts are undesirable signals that contaminate brain activity and are mostly of non-cerebral origin. Since the shape of neurological phenomenon is affected, artifacts may reduce the performance of BCI-based systems. artifacts may be classified into two major categories:

\begin{itemize}
 \item physiological artifacts
 \item non-physiological or technical artifacts
 \end{itemize}

Physiological artifacts are usually due to muscular, ocular and heart activity, known as electromyography (EMG), electrooculography (EOG), and electrocardiography (ECG) artifacts respectively. EMG artifacts, which imply typically large disturbances in brain signals, come from electrical activity caused by muscle contractions, which occur when patients are talking, chewing or swallowing. EOG artifacts are produced by blinking and other eye movements. Blinking makes generally high-amplitude patterns over brain signals in contrast to eye movements which produce low-frequency patterns. These electrical patterns are due to the potential difference between the cornea and the retina, as their respective charges are positive and negative. For that reason, the electric field around the eye changes when this dipole moves. EOG artifacts mostly affect the frontal area, because they are approximately attenuated according to the square of the distance. Finally, ECG artifacts, which reflect heart activity, introduce a rhythmic signal into brain activity.

Technical artifacts are mainly attributed to power-line noises or changes in electrode impedances, which can usually be avoided by proper filtering or shielding. Therefore, the BCI community focuses principally on physiological artifacts, given that their reduction during brain activity acquisition is a much more challenging issue than non-physiological artifact handling. Common methods for removing artifacts in EEG are linear filtering, linear combination and regression, ICA and PCA.

\section{Classification Algorithms}

The aim of the classification step in a BCI system is recognition of a user's intentions on the basis of a feature vector that characterizes the brain activity provided by the feature step~\cite{9}. Either regression or classification algorithms can be used to achieve this goal, but using classification algorithms is currently the most popular approach. Classification algorithms have traditionally been calibrated by users through supervised learning using a labelled data set. It is assumed that the classifier is able to detect the patterns of the brain signal recorded in online sessions with feedback. However, this assumption results in a reduction in the performance of BCI systems, because the brain signals are inherently non-stationary. The patterns observed in the experimental samples during calibration sessions may be different from those recorded during the online session. On the other hand, progressive mental training of the users or even changes in concentration, attentiveness, or motivation may affect the brain signals. Therefore, adaptive algorithms are essential for improving BCI accuracy. Adaptation to non-stationary signals is particularly necessary in asynchronous and non-invasive BCIs. K-Nearest Neighbour Classifier (k-NNC), Linear Discriminant Analysis (LDA), Support Vector Machine (SVM), Bayesian Statistical Classifier, Artificial Neural Network (ANN) were the main classification algorithms used.

\section{Applications}

Any device that can be connected to a computer or to a microcontroller could be controlled with a BCI. In practice however, the set of devices and applications that can be controlled with a BCI is limited. Some of the applications possible with current BCIs are described~\cite{1}.

\subsection{Spelling Devices}
 
Spelling devices allow severely disabled users to communicate with their environment by sequentially selecting symbols from the alphabet. One of the first spelling devices mentioned in the BCI literature is the P300 speller. A SCP based system in which the alphabet is split into two halves and subjects can select one halve by producing positive or negative SCPs. The selected halve is then again split into two halves and this process is repeated recursively until only one symbol remains. 

\subsection{Environment Control} 

Environment control systems allow controlling electrical appliances with BCI. P300 and SSVEP based synchronous BCI system are already developed. Development of asynchronous BCI systems is probably the most important research topic to advance the area of environment control systems.

\subsection{Wheelchair Control} 
Disabled subjects are almost always bound to wheelchairs. A BCI can potentially be used to steer a wheelchair. Steering a wheelchair is a complex task and wheelchair control has to be extremely reliable, the possible movements of the wheelchair are strongly constrained in current prototype systems. Both P300 and oscillatory process based wheelchair control systems have developed.

\subsection{Neuromotor Prostheses}
The idea underlying research on neuromotor prostheses is to use a BCI for controlling movement of limbs and to restore motor function in amputees. Different types of neuromotor prostheses can be envisioned depending on the information transfer rate a BCI provides. If neuronal ensemble activity is used as control signal, high information transfer rates are achieved and 3D robotic arms can be controlled. If an EEG based BCI is used, only simple control tasks can be accomplished.

\subsection{Gaming and Virtual Reality}
Besides the applications targeted towards disabled subjects, prototypes of gaming and virtual reality applications have been described. Examples for such applications are the control of a spaceship with oscillatory brain activity, the control of an animated character in an immersive 3D gaming environment with SSVEPs.



